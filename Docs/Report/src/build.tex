\documentclass[\main.tex]{subfiles}

\chapter{Construção}
\begin{singlespace}
\minitoc
\end{singlespace}
\vspace{20pt}

\section{Requisitos}

\subsection{Obrigatórios}
Para a realização deste projeto foram definidos alguns requisitos obrigatórios:
\begin{itemize}
    \item Suporte para ecrãs de diferentes dimensões (Telemóvel e Tablet);
    \item Uso das bibliotecas de suporte (Android Support Library);
    \item Uso de listas (RecyclerView e Adapters);
    \item Uso de base de dados (Room);
    \item Uso de operações assíncronas (AsyncTask/Thread/IntentService);
    \item Uso de notificações;    
    \item Uso de sensores de localização e disponibilização de informação em mapas;
    \item Uso das guidelines do material design;
    \item Integração com a API Zomato via pedidos REST (Retrofit).
\end{itemize}

\newpage
\subsection{Bonificação}
Para bonificar o projeto foram estabelecidos os seguintes requisitos:
\begin{itemize}
    \item Uso webservices adicionais via pedidos REST (Retrofit);
    \item Uso da biblioteca de deteção de atividades em Android para deteção do
    início ou fim de atividades;
    \item Exportar data de refeições realizadas para a API Google Calendar para
    registo de atividade no calendário pessoal;    
    \item Interação com elementos do Android (Contactos, Mensagens, Dialer, etc.).\\
\end{itemize}


\section{Implementação}

Para a realização desta aplicação foram utilizadas as seguintes ferramentas:
\begin{itemize}
    \item Java - linguagem de programação;
    \item AndroidStudio - IDE para desenvolvimento da aplicação;
    \item Visual Studio Code - editor de código para desenvolvimento dos documentos
    de suporte ao projeto;
    \item Gradle - controlo de dependências;
    \item Git - controlo de versões;
    \item GitHub - gestor do repositório do projeto.
\end{itemize}

\newpage