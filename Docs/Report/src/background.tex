\documentclass[\main.tex]{subfiles}

\chapter{Contextualização}
\begin{singlespace}
\minitoc
\end{singlespace}
\vspace{20pt}

\section{Restaurante}
Um restaurante é um Estabelecimento comercial destinado a confecionar e vender refeições,
onde normalmente são também servidos todo o tipo de bebidas.\\

\section{Aplicação móvel}
Uma aplicação móvel é um software desenvolvido com o intuito de ser instalado num
dispositivo móvel. Tem como propósito facilitar o quotidiano dos seus utilizadores.\\
\indent Originalmente as aplicações móveis foram criadas e classificadas como
ferramentas de apoio à produtividade e à recuperação de informação generalizada,
incluindo email, calendário, contatos, mercado de ações, informações meteorológicas,
etc. A crescente procura, a disponibilidade facilitada e a evolução , conduziu à rápida
expansão para outras categorias, como jogos, GPS, serviços de informação meteorológica,
serviços de acompanhamento de pedidos, compra de bilhetes, confirmações de presenças e
redes sociais.\\
\indent Nos dias de hoje, estão praticamente presentes em todo o lado incluindo nas áreas
da saúde, desporto, banca e negócios.\\

\section{Combinação}
Atualmente, onde o tempo é um recurso cada vez mais escasso, é indispensável que até a
mais simples tarefa de escolher o estabelecimento onde vamos realizar a nossa próxima
refeição seja feita de forma rápida e eficiente.\\
\indent Com a recente pandemia da Covid 19, torna-se também imprescindível que essa
escolha seja feita da forma mais acertada possível de forma a evitar o constrangimento de
nos deslocarmos para um local em vão, aumentando a probabilidade de sermos infetados.\\
\indent Deste modo, e para que os utilizadores da nossa aplicação não só se possam manter
o mais seguros possível, mas também desfrutar ao máximo da sua refeição, decidimos unir
os dois conceitos acima relatados resultando numa aplicação móvel para servir de suporte
aos utilizadores de restaurantes em Portugal continental.\\

\section{API da Zomato}
A \textit{\acrshort{api}} da Zomato permite obter informações relacionadas a mais de um
milhão e meio de restaurantes, bares, cafés, pubs e nightlife situados em dez mil cidades
em todo o mundo.\\
\indent Dentro destas informações são incluídos geralmente o nome e tipo do
estabelecimento, a sua localização, contactos, custo médio para duas pessoas, horários de
funcionamento, contactos, tipos de “cozinha” e uma avaliação.\\
\indent Além destas informações mais gerais, através da \textit{\acrshort{api}} também é
possível saber se o estabelecimento aceita cartão de crédito, se tem take-away e
entregas ao domicílio, se são servidas bebidas alcoólicas, se é provido de wi-fi, etc.\\

\newpage