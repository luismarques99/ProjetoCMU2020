\documentclass[\main.tex]{subfiles}

\chapter{Overview}
This chapter was planned to be an overview of the project so that each component
described here is separated by a page break to maintain a good consistency along the
whole document.
\newpage
\section{Web Application}
\begin{figure}[ht]
\centering
\includegraphics[width=\linewidth]{../private_assets/UML_UseCase_WebApp.png}
\caption{Web application Use Case diagram}
\end{figure}
Web application is intended to be simple and easy to use, so one of the main features
must be its responsiveness and usability. The user must be able to register with a few
steps as well as to login into the \gls{app}. Once the login is completed the user
can access its dashboard with all the passwords saved in its account. If the user so
desires it can select a group where many passwords are stored. Each group has its own
domain, promoting a better understanding and organization for the user to freely move
throughout the \gls{app}. Inside a group, it will be able to view all passwords in it
and apply certain operations, such as editing a group, deleting or creating a new one.
Clicking on a password box will show its information and able the opportunity to
ingress to the corresponding end-point. The same group operations can be apllied to
the passwords.

\newpage
\section{Web API}
\begin{figure}[ht]
\centering
\includegraphics[width=\linewidth]{../private_assets/UML_WebAPIArchitecture.png}
\caption{Web \acrshort{api} architecture}
\end{figure}
To begin, a \acrshort{http} request is made through an \acrshort{api} client such as
our web \gls{app}. The request is managed by a Controller and the access to the
\acrfull{db} is made by \acrshort{db} Context. To make the communication between
these components easier there is a Repository which allows read and write actions
between them. The main goal of the web \acrshort{api} is to be quite safe as we are
dealing with our customers passwords, so that each password needs to be encrypted
before going into the \acrshort{db}.\\
\indent \acrlong{db} information is defined by the Model's. To make everything even
more secure and ensure that there is no unintentional leak of information each Model
is mapped to a \acrfull{dto} to send in the \acrshort{http} response in
\acrshort{json} format

\newpage
\section{Mockups}
\begin{figure}[ht]
\centering
\includegraphics[height=7cm]{../private_assets/Mockup_SignIn.png}
\caption{Register a new user}
\vspace*{2cm}
\includegraphics[height=7cm]{../private_assets/Mockup_Login.png}
\caption{User login}
\end{figure}
\newpage
\vspace*{1cm}
\begin{figure}[ht]
\centering
\includegraphics[height=7cm]{../private_assets/Mockup_Home.png}
\caption{User's dashboard}
\vspace*{2cm}
\includegraphics[height=7cm]{../private_assets/Mockup_NewPassword.png}
\caption{Add a new password}
\end{figure}
\newpage
\vspace*{1cm}
\begin{figure}[ht]
\centering
\includegraphics[height=7cm]{../private_assets/Mockup_Password.png}
\caption{Password details}
\vspace*{2cm}
\includegraphics[height=7cm]{../private_assets/Mockup_NewGroup.png}
\caption{User's groups of passwords}
\end{figure}
\newpage