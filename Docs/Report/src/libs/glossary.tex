% \gls - lowercase singular
% \Gls - uppercase singular
% \glspl - lowercase plural
% \Glspl - uppercase plural

\newglossaryentry{apipt}
{
	name={Interface de Programação de Aplicações},
    description={Conjunto de rotinas e padrões estabelecidos por um software para a
    utilização das suas funcionalidades por aplicativos que não pretendem envolver-se
    em detalhes da implementação do software, mas apenas usar seus serviços}
}

\newglossaryentry{android}
{
	name=Android,
    description={Sistema operacional baseado no núcleo \gls{linux}, desenvolvido por
    um consórcio de desenvolvedores conhecido como Open Handset Alliance, sendo o
    principal colaborador o Google}
}

\newglossaryentry{linux}
{
	name=Linux,
    description={Termo popularmente empregado para se referir a sistemas operativos
    que utilizam o \gls{kernel} Linux}
}

\newglossaryentry{kernel}
{
	name=Kernel,
    description={Componente central do sistema operativo da maioria dos computadores;
    ele serve de ponte entre aplicações e o processamento real de dados feito a nível
    de hardware}
}