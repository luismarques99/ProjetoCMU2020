\documentclass[\main.tex]{subfiles}

\chapter{Introdução}
\begin{singlespace}
\minitoc
\end{singlespace}
\vspace{20pt}

\section{Âmbito}
Esto projeto foi realizado na época especial de 2020, dentro do âmbito de
\acrlong{cmu} da \acrlong{lei} na \acrlong{estg} - P.Porto.\\

\section{Definição do Problema}
A atual situação económica e social devido à pandemia COVID-19 forçou a adoção de novas medidas
e a adaptação de novas rotinas. A utilização de espaços de restauração ficou condicionada com
novas restrições de limites de ocupação e horários de funcionamento.\\
\indent Desta forma, torna-se útil pesquisar restaurantes e verificar o cardápio fornecido antes da
deslocação ao estabelecimento, de modo a evitar que o utilizador se dirija ao local e a comida
não seja do seu agrado, assim como descobrir novos restaurantes adaptados as necessidades de cada
cliente.\\

\section{Solução}
\indent De forma genérica pretende-se o desenvolvimento de uma aplicação para dispositivos móveis
que melhore a experiência da visita a restaurantes e registe os considerados “preferidos” pelo
utilizador, que estabelecimentos já frequentou e qual a pontuação fornecida de cumprimento de
normas de saúde pública relacionadas com o COVID-19.\\
\indent Deve ser permitido visualizar todos os restaurantes já visitados, os seus favoritos e todos
em que se pretende efetuar uma refeição considerando a distância do utilizador ao estabelecimento,
a classificação atribuída pelo utilizador (ou pela comunidade caso não exista uma classificação
específica do utilizador) e a frequência de utilização (número de vezes que o utilizador realizou
refeições nesse estabelecimento). A aplicação deve automaticamente detetar quando um utilizador se
encontra perto de um restaurante favorito, e automaticamente lançar uma notificação sugerindo uma
refeição nesse estabelecimento, mesmo quando a aplicação se encontrar em background.\\
\indent Para aquisição de informação de contexto sobre restaurantes deve ser considerada, entre
outras, a API Zomato. Outras APIs podem igualmente ser usadas em substituição (ex: google places)
desde que consigam dar resposta ao problema. Informação proveniente de outras fontes de dados ou
webservices podem também ser consideradas para enriquecimento da aplicação (ex: navegação,
meteorologia, etc).\\
\indent O histórico de refeições deve estar disponível para o utilizador e ser consultado pelo
utilizador da aplicação. Cada histórico deve apresentar de forma breve informação sobre o
estabelecimento, e outras informações relevantes como o tipo de refeição efetuada mais
frequentemente, distância média percorrida para que o utilizador se desloque.\\
\indent A pontuação fornecida (de 1 a 5) deve poder ser partilhada de forma anónima com todos os
utilizadores da aplicação a desenvolver (considere o uso de uma base de dados como firebase para
guardar as pontuações de cada restaurante). Um utilizador só pode pontuar um determinado
restaurante quando se encontrar perto dele (considerar um raio ou tolerância entre as coordenadas
do restaurante e a posição do utilizador).\\

\section{Objetivos}
Para este projeto foram estabelecidos alguns objetivos pelo professor:
\begin{itemize}
    \item Especificar e coordenar um projeto em grupo de pequena dimensão;
    \item Compreender e dominar os conhecimentos teóricos e práticos sobre desenvolvimento de
    aplicações móveis na plataforma Android;
    \item Adquirir competências com vista à resolução de problemas, nomeadamente através da
    pesquisa e utilização autónoma de conteúdos e ferramentas externas;
    \item Estimular o trabalho em equipa como elemento essencial do processo de aprendizagem
    individual.
\end{itemize}

\indent Para completar, nós também definimos alguns objetivos pessoais, para a realização deste
trabalho da melhor forma e com a garantia de que obteríamos o máximo de conhecimentos possíveis
sobre a plataforma \gls{android}:
\begin{itemize}
    \item Construir uma aplicação totalmente amiga do utilizador e fácil de compreender;
    \item Incorporar bastante responsividade na aplicação para que ela não quebre tão
    facilmente;
    \item Criar um design minimamente plausível e bonito da aplicação.\\
\end{itemize}

\section{Repositório}
https://github.com/LuisMarques99/ProjetoCMU2020

\newpage